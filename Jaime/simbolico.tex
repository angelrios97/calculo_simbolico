\documentclass[10pt,a4paper]{article}
\usepackage[utf8]{inputenc}
\usepackage{amsmath}
\usepackage{amsfonts}
\usepackage{amssymb}
\usepackage{amsthm}
\usepackage{graphicx}
\usepackage{mathtools}
\usepackage[left=3cm,right=3cm,top=2cm,bottom=2cm]{geometry}
\author{Ángel Ríos San Nicolás}
\title{Computación Simbólica}
\date{20 de noviembre de 2020}
\theoremstyle{plain}
\newtheorem{ejer}{Ejercicio}
\theoremstyle{definition}
\newtheorem*{sol}{Solución}
\begin{document}
\maketitle
\begin{ejer} En el caso $|f|=|g|=n$, si tomamos $k=\lceil\log_2(n) \rceil$, podemos añadir ceros a la representación de $f$ y $g$ hasta llegar a una representación $|f|=|g|=2^k$. Para esta representación, Karatsuba necesita hacer $\mathcal{O}((2^k)^{\log_2(3)})$. Compare, asintóticamente, las cantidades 
$$n^{\log_2(3)},\qquad (2^k)^{\log_2(3)}$$
\end{ejer}
\begin{sol}
Vamos a ver que $\mathcal{O}\left(n^{\log_2(3)}\right)=\mathcal{O}\left((2^k)^{\log_2(3)}\right)$ acotando los cocientes.

Como $\lceil\log_2(n)\rceil\leq \log_2(n)+1$ para todo $n\in \mathbb{N}$,
$$\frac{(2^k)^{\log_2(3)}}{n^{\log_2(3)}}
= \frac{\left(2^{\lceil \log_2(n)\rceil}\right)^{\log_2(3)}}{n^{\log_2(3)}}
 =  \frac{2^{\log_2(3)\lceil\log_2(n)\rceil}}{n^{\log_2(3)}}
\leq  \frac{2^{\log_2(3)(1+\log_2(n))}}{n^{\log_2(3)}}
 =  \frac{2^{\log_2(3)+\log_2(3)\log_2(n)}}{n^{\log_2(3)}}=$$ $$
 =  \frac{2^{\log_2(3)}2^{\log_2\left(n^{\log_2(3)}\right)}}{n^{\log_2(3)}}
=  3\frac{n^{\log_2(3)}}{n^{\log_2(3)}}=3.$$
Por tanto, existe un $N\in\mathbb{N}$ tal que para todo $n>N$, $(2^k)^{\log_2(3)}\leq 3n^{\log_2(3)}$ 
con lo que 
$$(2^k)^{\log_2(3)}\in\mathcal{O}\left(n^{\log_2(3)}\right).$$

Como $\log_2(n)\leq \lceil\log_2(n)\rceil$ para todo $n\in\mathbb{N}$,

$$\frac{n^{\log_2(3)}}{\left(2^{\lceil\log_2(n)\rceil}\right)^{\log_2(3)}}=\frac{n^{\log_2(3)}}{2^{\log_2(3)\lceil\log_2(n)\rceil}}\leq \frac{n^{\log_2(3)}}{2^{\log_2(3)\log_2(n)}}=\frac{n^{\log_2(3)}}{2^{\log_2\left(n^{\log_2(3)}\right)}}=\frac{n^{\log_2(3)}}{n^{\log_2(3)}}=1.$$
Por tanto, existe un $N\in\mathbb{N}$ tal que para todo $n>N$, $n^{\log_2(3)}\leq (2^k)^{\log_2(3)}$ y también $$n^{\log_2(3)}\in\mathcal{O}\left((2^k)^{\log_2(3)}\right).$$

Con esto hemos probado que 
$$\mathcal{O}\left((2^k)^{\log_2(3)}\right)=\mathcal{O}\left(n^{\log_2(3)}\right),$$
es decir, que las cantidades son iguales asintóticamente en términos de $\mathcal{O}$.
\end{sol}
\begin{ejer} Describa el método de multiplicación de Karatsuba para la multiplicación de enteros.
\end{ejer}
\begin{sol}
Sean $f,g\in\mathbb{Z}$. Podemos expresar $f,g$ en base $2$ de la forma
\begin{align*}
f &=a_0+a_12+a_32^2+\cdots+a_{n-1}2^{n-1} =[a_{n-1}\ldots a_0]\\
g &=b_0+b_12+b_32^2+\cdots+b_{n-1}2^{n-1} =[b_{n-1}\ldots b_0]
\end{align*}
con $a_i,b_i\in\{0,1\}$ para todo $i\in\{1,\ldots, n\}$ donde estamos suponiendo que $n=2^{k}$ con $k\in\mathbb{N}$ añadiendo ceros si es necesario.
\end{sol}
A partir de
\begin{align*}
f &= a_0+a_12+\cdots+a_{\frac{n}{2}-1}2^{\frac{n}{2}-1}+2^{\frac{n}{2}}\left(a_{\frac{n}{2}}+a_{\frac{n}{2}+1}2+\cdots+a_n2^{\frac{n}{2}}\right)\\
g &= b_0+b_12+\cdots+b_{\frac{n}{2}-1}2^{\frac{n}{2}-1}+2^{\frac{n}{2}}\left(b_{\frac{n}{2}}+b_{\frac{n}{2}+1}2+\cdots+b_n2^{\frac{n}{2}}\right),
\end{align*}
definimos 
\begin{align*}
f_0 &=a_0+a_12+\cdots+a_{\frac{n}{2}-1}2^{\frac{n}{2}-1}  &f_1=a_{\frac{n}{2}}+a_{\frac{n}{2}+1}2+\cdots+a_n2^{\frac{n}{2}}\\
g_0 &=b_0+b_12+\cdots+b_{\frac{n}{2}-1}2^{\frac{n}{2}-1}  &g_1=b_{\frac{n}{2}}+b_{\frac{n}{2}+1}2+\cdots+b_n2^{\frac{n}{2}},
\end{align*}
que son enteros de tamaño la mitad que los originales $f$ y $g$ y con los que se tiene $f=f_0+2^\frac{n}{2}f_1$ y $g=g_0+2^\frac{n}{2}g_1$. El producto es
$$fg=f_0g_0+2^\frac{n}{2}(f_0g_1+f_1g_0)+2^nf_1g_1.$$
Tomamos $u=f_0g_0$, $v=f_1g_1$ y $w=f_0g_0-f_0g_1-f_1g_0+f_1g_1=(f_0-f_1)(g_0-g_1)$, de manera que $f_0g_1+f_1g_0=w+u-v$ y podemos escribir

$$fg=u+2^\frac{n}{2}(u+v-w)+2^nv.$$

Sabemos que en las restas $f_0-f_1$ y $g_0-g_1$ no tenemos ningún dígito extra porque en valor absoluto $|f_0-f_1|\leq \max\{|f_0|,|f_1|\}$ con lo que el número de dígitos del resultado se puede tomar igual que el de las entradas. Podremos obtener resultados intermedios negativos, pero esto no afectará a la complejidad porque el número de dígitos es el mismo, solo hay que controlar el signo. Con esto sustiuimos el producto de dos enteros de $n>1$ dígitos en tres productos de enteros de $\frac{n}{2}$ dígitos y cuatro sumas de enteros de $n$ dígitos (porque al multiplicar dos números de $\frac{n}{2}$ dígitos obtenemos un número de a lo sumo $n$ dígitos). No contamos los productos por potencias de $2$ porque en base $2$ se corresponden simplemente a añadir ceros al final.

El algoritmo de Karatsuba consiste en aplicar este esquema de manera recursiva a los tres productos $f_0g_0$, $f_1g_1$ y $(f_0-f_1)(g_0-g_1)$ hasta que llegamos a productos de enteros de un dígito que sucede siempre porque $n$ es potencia de $2$.

Calculamos ahora el coste del algoritmo. Si $T(n)$ es el coste del algortimo de Karatsuba en número de operaciones bit para tamaño de la entrada en base $2$ igual a $n=2^k$, lo que acabamos de decir implica que $T(n)\leq 3T(\frac{n}{2})+4n$. Aplicando la recursión, obtenemos
\begin{align*}
T(n) &\leq 3T\left(\frac{n}{2}\right)+4n\leq \\
& \leq 3\left(3T\left(\frac{n}{4}\right)+4\frac{n}{2}\right)+4n=3^2T\left(\frac{n}{4}\right)+4n\left(\frac{3}{2}+1\right)\leq\\
& \leq 3^2\left(3T\left(\frac{n}{8}\right)+4\frac{n}{4}\right)+4n\frac{3}{2}= 3^3T\left(\frac{n}{8}\right)+4n\left(\left(\frac{3}{2}\right)^2+\frac{3}{2}+1\right)\leq \\
& \vdotswithin{\leq}\text{($k$ pasos)}\\
&\leq 3^kT(1)+4n\sum_{i=0}^{k-1}\left(\frac{3}{2}\right)^i=3^kT(1)+4n\frac{\left(\frac{3}{2}\right)^k-1}{\frac{3}{2}-1}=\\
& = 3^kT(1)+8\cdot 3^k-8n= (T(1)+8)3^k-8n\leq (T(1)+8)3^k.
\end{align*}
Por tanto, $T(n)\leq (T(1)+8)3^k$ donde $T(1)+8$ es una constante con lo que $T(n)\in\mathcal{O}\left(3^k\right)$, pero como
$$3^k=3^{\log_2(n)}=2^{\log_2(3)\log_2(n)}=2^{\log_2\left(n^{\log_2(3)}\right)}=n^{\log_2(3)},$$
el coste del algoritmo de Karatsuba es $\mathcal{O}\left(n^{\log_2(3)}\right)$ en el número de operaciones bit.
\begin{ejer} Busque en qué consiste el método de Horner y calcule el número de operaciones que realice. Compare el resultado con el lema anterior.
\end{ejer}
\begin{sol}
El método de Horner es un algoritmo para evaluar polinomios univariados.

Si $n\in\mathbb{N}$, $f\in\mathbb{K}[X]$, $f=a_0+a_1X+a_2X^2+\cdots+a_nX^n$ y $x\in \mathbb{K}$. Podemos ir sacando factor común a $X$ progresivamente manteniendo la igualdad de la siguiente manera:

\begin{align*}
f &= a_0+a_1X+a_2X^2+\cdots+a_nX^n\\
&= a_0+X(a_1+a_2X+\cdots+a_nX^n)\\
&= a_0+X(a_1+X(a_2+a_3X+\cdots+a_nX^n))\\
&\vdotswithin{=}   \\
& = a_0+X(a_1+X(a_2+X(\cdots+X(a_{n-1}+Xa_n)\cdots))) 
.\end{align*}
Observando esto definimos, para cada $k\in \{0,\ldots,n\}$, el polinomio $$b_k=a_k+X(a_{k+2}+X(a_{k+3}+X(\cdots +X(a_{n-1}+Xa_n)\cdots))).$$
Claramente se cumplen:
\begin{align*}
b_0 &= a_n\\
b_1 &= a_{n-1}+Xb_n=a_{n-1}+Xa_n\\
b_2 &= a_{n-2}+Xb_{n-1}=a_{n-2}+X(a_{n-1}+Xb_n)=a_{n-2}+X(a_{n-1}+Xa_n)\\
&\vdotswithin{=}\\
b_{n-2} &=a_2+Xb_3=a_2+X(a_3+Xb_4)=\cdots=a_2+X(a_3+X(\cdots+X(a_{n-1}+Xa_n)\cdots))\\
b_{n-1} &=a_1+Xb_2=a_1+X(a_2+Xb_3)=\cdots=a_1+X(a_2+X(\cdots +X(a_{n-1}+Xa_n)\cdots))\\
b_n &=a_0+Xb_1=a_0+X(a_1+Xb_2)=\cdots=a_0+X(a_1+X(\cdots +X(a_{n-1}+Xa_n)\cdots))=f
\end{align*}
El algoritmo de Horner consiste en calcular la evaluación $f\left(X=x\right)$ en $n+1$ pasos calculando en orden $b_0(x),b_1(x)\ldots, b_n(x)=f(x)$ donde, aprovechando los cálculos anteriores, reducimos el número de operaciones.

Vamos a calcular el número de sumas y productos en $\mathbb{K}$.
En $b_0(x)$ no hacemos ni productos ni sumas. Para cada $k\in\{1,\ldots, n\}$, en $b_k(x)$ hacemos el producto $xb_{k-1}(x)$ y la suma $a_k+xb_{k-1}(x)$ donde $b_{k-1}(x)$ había sido calculado ya en el paso anterior. En total, el método de Horner evalúa el polinomio haciendo $n$ sumas y $n$ productos. El coste del algoritmo es $\mathcal{O}(n)$ en el número de operaciones en $\mathbb{K}$.

Utilizando el método de Horner, hacemos $n-1$ productos menos respecto al algoritmo del lema anterior en el que se calculan las potencias $a^m$ con $m\in\{0,\ldots, n\}$ aprovechando las anteriores lo que en total son $n-1$ productos a los que hay que sumar los $n$ productos por los coeficientes del polinomio, es decir, en total $2n-1$ productos. El número de sumas es el mismo.

\end{sol}
\end{document}